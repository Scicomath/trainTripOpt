\documentclass[UTF8]{ctexart}
\usepackage{graphicx}
%\usepackage{bbding}
\usepackage{url}
\usepackage{booktabs}
\usepackage{amsmath,amssymb}

\begin{document}
\title{铁路最优路线问题}
\author{艾鑫 李露 黄彬}
\maketitle

\section{摘要}
这里是摘要
\newpage

\section{问题重述}
\subsection{问题说明}
铁路既是社会经济发展的重要载体之一,同时又为社会经济发展创造了前提条件.近几年来,在全社会客运量稳步上升的同时,长期以来铁路承运了大量旅客.相对于其他的运输方式铁路具有时间准确性高、运输能力大、运行比较平稳、安全性高等优点.同时火车也成为了旅途的首选交通运输工具.

虽然目前铁路网络已经比较发达,但是仍然有很多地方之间并没有直接到达的铁路.并且在节假日期间,一些热门路线的火车票总是一票难求.在这种情况下,需要考虑换乘,即先从乘车站到换乘站,再从换乘站到目的站.

\subsection{需要解决的问题}
\begin{enumerate}
  \item 给出任意两个站点之间的最优铁路路线问题的一般数学模型和算法.若两个站点之间有直达列车,需要考虑直达列车票已售罄情况下最优的换乘方案.根据附录数据,利用你们的模型和算法求出一下起点到终点的最优路线:丹东→宜昌、天津→拉萨、白城→青岛.
  \item 假设你从打算从宜昌出发乘火车到上海、南京、杭州、苏州、无锡旅游最后回到宜昌,请建立相关数学模型,给出整个行程的最优路线.
\end{enumerate}

\section{模型假设}
\begin{description}
  \item[假设1] 假设所有列车都不晚点,准时到达.
  \item[假设2] 不考虑换乘所花的时间.
  \item[假设3] 从起点乘车时所等车的时间忽略不计.
  \item[假设4] 换乘次数不超过1次为可接受的换乘次数.
\end{description}

\section{符号说明}
\begin{table}[h]
  \centering
  \begin{tabular}{cl}
     \toprule
     % after \\: \hline or \cline{col1-col2} \cline{col3-col4} ...
     符号 & 符号说明 \\
     \midrule
     $V$ & 站点构成的集合 \\
     $A$ & 任意两个站点之间连通的路线构成的集合 \\
     $n_0$ & 起始站点 \\
     $n_k$ & 终到站点 \\
     $C(v_i,v_j)$ & 时间权值集合 \\
     $W(v_i,v_j)$ & 费用权值集合 \\
     $f(v_i,v_j)$ & 任意两个站点之间能否连通构成的邻接矩阵 \\
     $M_{ij}$ & 换乘次数 \\
     $W(n_0,n)$ & 乘车总费用 \\
     $T(n_0,n)$ & 乘车总时间 \\
     \bottomrule
   \end{tabular}
  %\caption{}\label{}
\end{table}

\section{问题分析}
本文要解决的是铁路换乘最佳路线的选择问题.针对文中提出的问题,在综合考虑换乘次数,乘车时间和乘车费用的前提下,给出任意两个站点之间的最优换乘路线.对问题的具体分析如下:

\subsection{问题一的分析}
本题要求给出任意两个站点之间的最优铁路路线问题的一般数学模型和算法,并根据此算法求出题中给出的3对始终站之间的最优路线.在选择乘车路线时一般考虑三个因素即始终站之间的换乘次数、乘车时间和乘车费用.因为铁路列车的交通网络比较复杂,列车车次繁多,很难掌握各列车的发车时间和所经站点,为了减少换乘所带来的麻烦,据此把换乘次数最少作为选择最优路线的主要目标,由于主观原因,每个人的经济状况和时间空余安排不同,在考虑乘车时间和乘车费用时的主次有所不同,因此将这两个目标函数的优先级作以变换,为此列出了以换乘次数最少、乘车时间最少和乘车费用最少为目标的多目标函数,要求换乘时从起点到达换乘站时的时间要小于换乘列车的出发时间,换乘的站点之间要有路线存在以保证能够换乘成功.运用分层序列法和广度优先搜索算法,利用MATLAB编程可求出优先考虑乘车时间和乘车费用两种情况下的最优路线.
\subsection{问题二的分析}
本题要求建立相关数学模型,求出从宜昌出发乘火车到上海、南京、杭州、苏州、无锡最后回到宜昌的最优旅游路线.此题仍是一个多目标最优化问题,将这六个城市作为六个节点,构成一个带权有向图,且乘车站点只能在这六个点之间.列出了以乘车时间最少和乘车费用最少为目标的多目标函数,约束条件与问题一有所不同,即规定每个车站只有一条进去的路线和一条出去的路线;除起点和终点外,各个车站之间不能构成封闭的环.由于主观原因,将这两个目标函数的优先级作以变换,运用分层序列法和广度优先搜索算法,可以求出优先考虑乘车时间和乘车费用两种情况下的最优路线.

\section{数据处理与分析}
\subsection{车次、站点和车次类型的统计}
由于附录中的数据量庞大,车次和站点的冗杂性比较高,利用MATLAB编程处理附录中的数据可以得到:全国共有2867个站点、4683个车次,列车类型共有11中,其中城际高速只存在于北京到天津、上海到金山卫之间,只有少数地区可供选择,且具有速度高价格贵等特点.

各种类型列车的列车车次数如下图所示:

从上图可以看出,新空快速、高速动车、动车组的车次数明显高于其他列车次数,我国目前运营的列车主要是快速空调列车、动车和高速动车,因此选择出游列车乘车方式时,大多数应该是选择这三种.

\subsection{任意两个站点间票价的计算}
附录的数据中只给出了起点到各个站点的票价,任意两个站点之间的票价并未给出,根据铁道部公布的信息,票价包括三部分:基本客票票价、附加票票价和其他(保险费、客票发展金等),根据铁道部公布的票价计算方案,可得到各车次的硬座车票价格计算公式如下:

变量的定义:两站之间的里程为$x$,普通火车基本票率为$\eta_0$,动车组票率为$\eta_1$,高速动车组票率为$\eta_2$,区段起点为$m_i$,区段终点为$n_i$,递远递减率为$\varepsilon_i$,保险票率为$\sigma$,客票发展金为$w$,空调票率为$\phi$,从起始站点到第$p$个站点的票价为$w_p$,从起始站点到第 个站点的票价为$w_q$.

\begin{table}[h]
\centering
\begin{tabular}{cl}
\toprule
列车类型 & 计算公式 \\
\midrule
  动车组 & $W_1 = \eta_1 x$ \\
  高速动车 & $W_2 = \eta_2 x$ \\
  新空直达 & $W_{3} = y_n - y_{n-1}$ \\
  城际高速 & $W_{4} = y_n - y_{n-1}$ \\
  普客 & $W_5 = \eta_0 (\sum_{i=1}^k (n_i - m_i)\varepsilon_i + (x - n_k))(1+\sigma)+w$ \\
  普快 & $W_6 = \eta_0 (\sum_{i=1}^k (n_i - m_i)\varepsilon_i + (x - n_k))(1+\sigma+\lambda)+w$ \\
  快速 & $W_7 = \eta_0 (\sum_{i=1}^k (n_i - m_i)\varepsilon_i + (x - n_k))(1+\sigma+2\lambda)+w$ \\
  新空普客 & $W_8 = \eta_0 (\sum_{i=1}^k (n_i - m_i)\varepsilon_i + (x - n_k))(1+\sigma+\phi)+w$ \\
  新空普快 & $W_9 = \eta_0 (\sum_{i=1}^k (n_i - m_i)\varepsilon_i + (x - n_k))(1+\sigma+\lambda+\phi)+w$ \\
  新空快速 & $W_{10} = \eta_0 (\sum_{i=1}^k (n_i - m_i)\varepsilon_i + (x - n_k))(1+\sigma+2\lambda+\phi)+w$ \\
  新空特快 & $W_{11} = \eta_0 (\sum_{i=1}^k (n_i - m_i)\varepsilon_i + (x - n_k))(1+\sigma+2\lambda+\phi)+w$ \\
  \bottomrule
\end{tabular}
%\caption{}
\end{table}

通过编程计算得到每个车次的列车任意两点之间的票价.

\section{问题一模型的建立与求解}
针对问题一,要求用给出的模型和算法求出3对始终站之间的最优路线,首先建立衡量最优路线的三个指标,然后运用分层序列法和广度优先搜索算法求出最优路线.

\subsection{模型的准备}
分层序列法:分层序列法是将目标函数分清主次,按其重要程度排序,然后依次对各目标函数求最优点.后者应该在前者的最优点集合域内寻优.假设 最重要,对其求解:
$$\min f_1(x), \quad x \in D$$

求得其最优值为$f_1^*$.在可行域$D$中,$f_1(x) \leq f_1^*$的区域称为最优点集合域,表示为$D_1 = \{x|f_1(x) \leq f_1^*\}$.

然后,在$D_1$内求第二个分目标函数的最优值:
$$\min f_2(x), \quad x \in D_1$$

求得其最优值为$f_2^*$.$f_2(x)$的最优点集合域$D_2 = \{f_1(x)\leq f_1^*, f_2(x)\leq f_2^*\}$. 然后在$D_2$内求$f_3(x)$的最优值.依次继续进行下去,最后求得$f_q^*$,对应的设计点为多目标优化问题的最优点$x^*$.

\subsection{模型的建立}
将整个铁路线路网建立成一个带权有向图$G$,将站点作为节点,列车通过两个站点之间的线路作为边,节点集$V = \{v_1,v_2,\cdots,v_n\}$,边集$A = \{a_1,a_2,\cdots,a_m\}$,构成带权有向图$G = (N,A)$,记有向图中由节点$v_i$、$v_j$确定的边为$a(v_i,v_j)$.

定义邻接矩阵$F$,若存在同一辆列车经过$v_i$和$v_j$,则$v_i$和$v_j$邻接,定义邻接矩阵如下:
$$F = (f(v_i,v_j))_{n\times n} \in \{0,1\}^{n \times n}$$
\[f(v_i,v_j) = \begin{cases}
1, & (v_i,v_j) \in A \\
0, & (v_i,v_j) \notin A
\end{cases}\]

定义每条边$a(v_i,v_j)$都有若干个行驶时间权值、发车时间权值和乘车费用权值,行驶时间权值为列车从$v_i$行驶到$v_j$的行驶时间,发车时间权值为列车从$v_i$行驶到$v_j$的列车的出发时间,乘车费用权值为列车从$v_i$行驶到$v_j$的费用。记时间权值为$c^k(v_i,v_j)$,发车时间权值为$t^k(v_i,v_j)$,费用权值为$w^k(v_i,v_j)$,若有$r$个车次经过站节点$v_i$和$v_j$,则$a(v_i,v_j)$有$r$个时间权值和费用权值,记$a(v_i,v_j)$得时间权值集合和费用权值集合分别为 $C(v_i,v_j)$、$W(v_i,v_j)$,有
\[C(v_i,v_j) = \{c^1(v_i,v_j),c^2(v_i,v_j),\cdots,c^r(v_i,v_j)\}\]
\[T(v_i,v_j) = \{t^1(v_i,v_j),t^2(v_i,v_j),\cdots,t^r(v_i,v_j)\}\]
\[W(v_i,v_j) = \{w^1(v_i,v_j),w^2(v_i,v_j), \cdots, w^r(v_i,v_j)\}\]

 将从节点$n_0$到节点$n_k$的乘车方案表示为:
  \[P = \{n_0^{x_0}, n_1^{x_1}, n_2^{x_2}, \cdots,n_m^{x_m},n_k^{x_k}\}\]
 其中$n_1,n_2,\cdots,n_m$为中转节点, $x_0$表示边$a(n_0,n_1)$的行驶时间权值、发车时间权值和费用权值分别取$c^{x_k}(n_0,n_1)$、$t^{x_k}(n_0,n_1)$和$w^{x_k}(n_0,n_1)$,同理$x_k$表示边$a(n_k,n_{k+1})$的行驶时间权值、发车时间权值和费用权值分别取$c^{x_k}(n_k,n_{k+1})$、$t^{x_k}(n_k,n_{k+1})$和$w^{x_k}(n_k,n_{k+1})$。
 
 \subsubsection{目标函数}
 换乘次数$m$最小,在乘车方案 中换乘次数等于中间节点的数目:
 \[\min M_{ij} = m\]
 
乘车所用时间最少,乘车时间=所换乘列车的发车时间—从始点出发的列车的发车时间+乘坐换乘列车的行驶时间:
\[\min T(n_0,n) = \min \sum_{ k=0}^{m} t^{x_k}(n_k,n_{k+1})  = t^{x_m}(n_m,n)-t^{x_m}(n_0,n_1)\]

乘车所花费用最少,乘车费用等于从起始站到换乘站之间的车费加上从换乘站到终点站之间的车费:
\[\min W(n_0,n) = \min \sum_{k=0}^{m} w^{x_k} (n_k,n_{k+1})\]

\subsubsection{约束条件}
换乘时从起点到达换乘站时的时间要小于换乘列车的出发时间,以保证能够顺利换乘车辆:
\[
t^{x_k}_k(n_k,n_{k+1}) + c^{k+1}_k (n_k+n_{k+1}) \leq t^{x_{k+1}}_{k+1}(n_{k+1},n_{n+2})
\]

换乘的站点之间要有路线存在,以保证能够换乘成功:
\[
\forall n_k, n_{k+1} \quad f(n_k,n_{k+1})=1 
\]

综上所述得到问题一的多目标优化模型如下:
\begin{equation}
\begin{array}{rrclcl}
\displaystyle \min_{x} & \multicolumn{3}{l}{c^T x} \\
\textrm{s.t.} & A x & \leq & b \\
&\displaystyle \sum_{i=0}^{n} x_i & = & 1 \\
& x_j & \geq & 0 & & \forall j \in N \\
\end{array}
\end{equation}
\begin{thebibliography}{10}
\bibitem{1} 全国大学生数学建模竞赛组委会, \textbf{高教社杯全国大学生数学建模竞赛论文格式规范}, 北京,2009
\bibitem{CCT}张林波,\textbf{新版CCT的说明},2006
\bibitem{2} \url{http://www.chinatex.org}
\bibitem{3}Tobias Oetiker, \textbf{一份不太简短的\LaTeXe 介绍},{下载地址如下}\\
    \url{http://tug.ctan.org/tex-archive/info/lshort/chinese/lshort-zh-cn.pdf}
\bibitem{4} Alpha Huang, \textbf{latex-notes-zh-cn},下载地址如下\\
    \url{http://mirrors.xmu.edu.cn/CTAN/info/latex-notes-zh-cn/}
\bibitem{lf}M.R.C. van Dongen,\textbf{\LaTeX-and-Friends},下载地址如下\\
    \url{http://csweb.ucc.ie/~dongen/LaTeX-and-Friends.pdf}
\bibitem{figure}Keith Reckdahl,\textbf{Using Import graphics in \LaTeXe}, 1997\\
王磊翻译,\textbf{\LaTeXe 插图指南},2000,下载地址如下\\
\url{http://math.ecnu.edu.cn/~latex/docs/Chs_doc/Graphics3.pdf}
\bibitem{HM}Addison Wesley,\textbf{Higher Mathematics}, 下载地址如下\\ \url{http://media.cism.it/attachments/ch8.pdf}
\bibitem{11} 全国大学生数学建模竞赛组委会, \textbf{高教社杯全国大学生数学建模竞赛论文格式规范}, 北京,2009
\bibitem{12} 全国大学生数学建模竞赛组委会, \textbf{高教社杯全国大学生数学建模竞赛论文格式规范}, 北京,2009
\bibitem{13} 全国大学生数学建模竞赛组委会, \textbf{高教社杯全国大学生数学建模竞赛论文格式规范}, 北京,2009

\bibitem{14} 全国大学生数学建模竞赛组委会, \textbf{高教社杯全国大学生数学建模竞赛论文格式规范}, 北京,2009
\end{thebibliography}




\end{document}
